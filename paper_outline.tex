%\documentclass[aps,floatfix,11pt,twocolumn]{revtex4-1}
\documentclass[aps,floatfix,11pt]{revtex4-1}
\usepackage{bm}%bold math
\usepackage{graphicx}
\usepackage{amsmath}
\usepackage{amssymb}
\usepackage{color}
\usepackage{setspace}
\usepackage{epstopdf}
\usepackage{scalerel}
\epstopdfsetup{update} % only regenerate pdf files when eps file is newer
\linespread{1}
\usepackage[export]{adjustbox}

\begin{document}
%%%%%%%%%%%%%%%%%%%%%%%%%%%%%%%%%%%%%%%%%%%%%%%%%%%%%%%%%%%%%%%%%%%%%%%%%%%%%%%%%%%%%%%%%%%%%%%
%%%%%%%%%%%%%%%%%%%%%%%%%%%%%%%%%%%%%%%%%%%%%%%%%%%%%%%%%%%%%%%%%%%%%%%%%%%%%%%%%%%%%%%%%%%%%%%
\section{Introduction}

    \begin{itemize}
        \item Topological Order that can not be described with Landau symmetry breaking.
        \item Robust ground state degeneracy.
        \item Fractional excitations.
        \item Toric code is a well studied exactly solvable model exhibitying topological order.
        \item Dimer models $\rightarrow$ more physical and some display topological order.
        \item Dimer models must be studied numerically.
    \end{itemize}

    \subsection{In this paper we... }
        \begin{itemize}
            \item In this paper we study the quantum dimer pentamer model (QDPM) on the square
                lattice at the RK point.
            \item Believe it may exhibit $Z_3$ topological order.
            \item Different than $Z_3$ toric code because only Ising degrees of freedom on the links.
            \item Study dimer correlations showing a liquid dimer pentamer phase at RK point with
                exponentially decaying dimer correlations.
            \item Estimate the dimer and vison gap.
        \end{itemize}


\section{Backround}

    \subsection{$Z_2$ Toric Code on Square Lattice}
        \begin{itemize}
        \item Hamiltonian.
        \item All terms commute $\rightarrow$ can solve exactly.
        \item Ground state is uniquely determined. 
        \item String operators.
        \item On torus conserved numbers, winding numbers $\rightarrow$ write in terms of string
            operators.
        \item 4 fold degenerate ground state.
        \item broken string operators give you electric and magnetic excitations.
        \item Exact local gauge symmetry on square lattice.
        \item $Z_2$ topological order $\rightarrow$ known exactly. 
        \end{itemize}

    \subsection{$Z_3$ Toric Code}
        \subsubsection{Hamiltonian Gauge symmetry }
            \begin{itemize}
                \item $Z_3$ algebra.
                \item Hamiltonian. Write the Hamiltonian in terms of $P$ and $Q$ operators. 
                \item \textcolor{red}{Figure} of vertex and plaquette labeling.
                \item Hamiltonian has local symmetry. 
                \item Specifically Hamiltonian obeys this local $Z_3$ gauge symmetry. Define the gauge operator. 
                \item Discus gauge symmetry connection to the Hilbert space.
                %\item express gauge operator in terms of electric flux $G_i =
                %    exp{[\frac{i2\pi}{3}\sum_{l\in v}\mathrm{sign}(l)E_l]}$
            \end{itemize}
        \subsubsection{String operators}
            \begin{itemize}
                \item String net $E$, $Q$ and $P$ operators and how they operate on links with three degrees of
                    freedom.
                \item Define string operators.
            \end{itemize}
    
        \subsubsection{properties of $Z_3$ string operators}
            \begin{itemize}
                \item \textcolor{red}{Figure} of the magnetic and electric string operator 
                    acting on a configuration (open ends).
                \item Use figure to highlight the fact that the $Z_3$ electric string operator with open ends
                    has a positive charge at one end and a negative charge at the other.
                \item Visons live at the ends of magnetic string with open ends.
                \item $Z_3$ vison $\rightarrow$ three values.
                \item Discuss $Z_3$ vison phase change when a closed electric string operator encloses a vison.
                \item Visons are the most ``physical'' excitations $\rightarrow$ explain. 
            \end{itemize}

        \subsubsection{Winding Numbers}
                \begin{itemize}
                    \item Winding operator in terms of magnetic string operator $\rightarrow$ commutes
                        with Hamiltonian.
                    \item Changing the topological sector with a non-trivial electric
                        string.
                    \item On a torus 9-fold degenerate ground state, each corresponding to a
                        different topological sector.
                \end{itemize}

    \subsection{QDM on Square lattice}
        \begin{itemize}
            \item Hamiltonian.
            \item RK point and Hilbert space.
            \item Dimer correlation's algebraic decay.
            \item Gapless.
            \item Discus $U(1)$ symmetry of QDM on square lattice.
            \item Winding number.
            \item Dimers as flux lines.
            \item Introduction of dynamical charges reduces $U(1) \rightarrow Z_N$ where $N$ is
                charge of itinerant objects.
        \end{itemize}


\clearpage
%%%%%%%%%%%%%%%%%%%%%%%%%%%%%%%%%%%%%%%%%%%%%%%%%%%%%%%%%%%%%%%%%%%%%%%%%%%%%%%%%%%%%%%%%%%%%%%
%%%%%%%%%%%%%%%%%%%%%%%%%%%%%%%%%%%%%%%%%%%%%%%%%%%%%%%%%%%%%%%%%%%%%%%%%%%%%%%%%%%%%%%%%%%%%%%
\section{The Quantum Dimer Pentamer Model}

    \subsection{How to get local $Z_3$ symmetry in square lattice dimer model}
        \begin{itemize}
            \item Define pentamer.
            \item When charge 3 is introduced pentamers are the flux from the vertices with
                additional charge 3.
            \item $U(1) \rightarrow Z_3$ through the relaxation of constraints.
            \item \textit{Exact} local gauge symmetry.
        \end{itemize}

    \subsection{QDPM Hamiltonian}
        \begin{itemize}
            \item QDM Hamiltonian + pentamer terms.
            \item Like the QDM pentamer terms will consist of the smallest local moves that conserve
                the constraints.
            \item Introduce the pentamer terms. \textcolor{red}{Figure} of pentamer dynamics.
            \item We study this model at the RK point $t/v = 1$ where the wave function is just the
                equal weighted superposition of all classical dimer pentamer configurations.
        \end{itemize}

    \subsection{Winding operators}
        \begin{itemize}
            \item Make conection with $Z_3$ toric code. Can use same string operators $\rightarrow$
                all QDPM configurations can be written as $Z_3$ toric code configurations (but not
                the other way around).
            \item QDPM winding operator in terms of magnetic string operator $\rightarrow$ commutes
                with Hamiltonian.
            \item QDPM simplified winding operator (mod 3).
            \item In QDPM model the QDM winding operator is no longer conserved and does not commute with the
                Hamiltonian.
            \item In QDPM on a torus 9-fold degenerate ground state, each corresponding to a
                different topological sector.
        \end{itemize}

\clearpage
%%%%%%%%%%%%%%%%%%%%%%%%%%%%%%%%%%%%%%%%%%%%%%%%%%%%%%%%%%%%%%%%%%%%%%%%%%%%%%%%%%%%%%%%%%%%%%%
%%%%%%%%%%%%%%%%%%%%%%%%%%%%%%%%%%%%%%%%%%%%%%%%%%%%%%%%%%%%%%%%%%%%%%%%%%%%%%%%%%%%%%%%%%%%%%%
\section{Dimer Correlation Results}
    \subsection{Correlation functions}
        \begin{itemize}
            \item By measuring the dimer correlations between the link at the origin and all other
                links on the lattice we find no obvious broken symmetries.
            \item List all of the correlation functions measured. Such as: parallel dimers along
                horizontal and vertical directions, perpendicular dimers along horizontal and
                vertical directions ...
            \item \textcolor{red}{Figure} of the correlation functions. If possible, one figure
                containing all of the important ones. Log linear plot with a linear linear inset.
        \end{itemize}

    \subsection{Liquid}
        \begin{itemize}
            \item The exponential decaying correlations mean a liquid state with a finite gap.
            \item The correlation length is...
        \end{itemize}

    \subsection{Comparison to other models}
        \begin{itemize}
            \item Take the parallel dimer correlation function and compare these results in the QDM,
                the QDPM, and the toric code.
            \item \textcolor{red}{Figure} with the dimer correlations of the three models listed
                above.
        \end{itemize}

\clearpage
%%%%%%%%%%%%%%%%%%%%%%%%%%%%%%%%%%%%%%%%%%%%%%%%%%%%%%%%%%%%%%%%%%%%%%%%%%%%%%%%%%%%%%%%%%%%%%%
%%%%%%%%%%%%%%%%%%%%%%%%%%%%%%%%%%%%%%%%%%%%%%%%%%%%%%%%%%%%%%%%%%%%%%%%%%%%%%%%%%%%%%%%%%%%%%%
\section{Estimating the Gap}
    \subsection{Imaginary time vison correlations}
        \begin{itemize}
            \item The $Z_3$ Vison gap should be the smallest if there is $Z_3$ topological order.
            \item We can estimate the vison gap by calculating the imaginary time correlation
                functions - present correlation function formally.
            \item Should we discuss the subtleties of the imaginary time vison correlation function?
            \item We investigate three particular points $\rightarrow$ origin, $(L/2,0)$, $(L/2,L/2)$.
            \item \textcolor{red}{Figure} containing the $Z_3$ vison imaginary time correlations at all three
                of the points listed above.
            \item Estimate of the vison gap is ...
            \item Compare single and double vison correlations.
        \end{itemize}

    \subsection{Imaginary time dimer correlations}
        \begin{itemize}
            \item Same thing for the dimer gap $\rightarrow$ define correlation function.
            \item investigate the same 3 points.
            \item \textcolor{red}{Figure} of the imaginary time dimer correlations.
            \item Estimate of the dimer gap is ...
        \end{itemize}

\clearpage
%%%%%%%%%%%%%%%%%%%%%%%%%%%%%%%%%%%%%%%%%%%%%%%%%%%%%%%%%%%%%%%%%%%%%%%%%%%%%%%%%%%%%%%%%%%%%%%
%%%%%%%%%%%%%%%%%%%%%%%%%%%%%%%%%%%%%%%%%%%%%%%%%%%%%%%%%%%%%%%%%%%%%%%%%%%%%%%%%%%%%%%%%%%%%%%
\section{Discussion}
    \begin{itemize}
        \item `` We present the QDPM ...''
        \item We show the string-net representation of the QDPM, introduce string operators and
            discus $Z_3$ excitations
        \item By calculating the dimer correlations we show QDPM is in liquid state at RK point.
        \item By calculating the imaginary time dimer and vison correlation functions we show ...
            hopefully show vison gap is smaller and thus this is indicative of $Z_3$ topological
            order.
    \end{itemize}

\end{document}
